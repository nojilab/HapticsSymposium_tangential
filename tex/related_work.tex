\section{RELATED WORK}
\subsection{Grounded-type or Wearable-Type Pin-array Display}

% pin arrayにも設置型と装着型の2種類ある。
There are two types of pin-array display: grounded-type or wearable-type.
The mechanically grounded pin-array display could present robust haptic cues using grounded forces with users~\cite{Shimizu1993, Howe, Leithinger:2010:RSA:1709886.1709928}.
However, When users move the hand or finger to any position around users, these devices cannot display shapes.
In other words, the workspace and portability are constrained.
A large interactive mobile workspace may be useful for the exploration of virtual spaces.

% 近年は装着型が多く開発されている
Recently, more haptic system designs have started appearing with wearability in mind, and in this context wearable, pin-array displays have been developed~\cite{Koo2008, Kim2009}.
Thus, the present study focuses on these finger-mounted, wearable pin-array displays.

\subsection{Presenting Force Distribution using Pin-Arrays}

% 既存ピンアレイでは法線方向のみの提示が実現されてきた
Many researchers have developed pin-array display which can present users with force distribution feedback~\cite{Moy2000,Velazquez2005,Sarakoglou2005,Kim2009,Jang:2016:HED:2858036.2858264,Benko:2016:NTH:2984511.2984526}.
In order to provide more cutaneous cues, researchers have attempted to increase the denser pin-array display.
For example, Kim et al.~\cite{Kim2009} developed a wearable display composed of a $4\times8$ pin array on the fingertip.
The diameter of the pin was 0.5 mm, and the pins were arranged in a 1.5 mm interval.
The main focus of these previous studies existed on the simple structure of pin and actuators for denser pin arrangements.
In contrast, the increase in the degree of freedom of force feedback has not been paid attention to.
For example, most of them were able to present only normal force distributions.


\subsection{Presenting Force in Various Direction}

% でも接線方向も大事
Although the presentation of force in the normal direction is important in communicating contact with the object, in order to realize the more natural interaction close to reality with the virtual object, it is also necessary to present the force in shear direction caused by friction.

% 接線方向にも力を提示できるディスプレイが開発されている。
% しかし一方向の力しか提示できず力の分布は提示できない
There were studies that developed display presenting shear force feedback.
For example, Minamizawa et al.~\cite{Minamizawa:2007:GGW:1278280.1278289} developed a method of laterally displacing the pad pressed against the finger by the force of the actuator in two-dimension. 
The work by ~\cite{6636291,Schorr:2017:FTD:3025453.3025744} extended it to three-dimension and their display moved the pad three degrees of freedom.
However, it is impossible to reproduce force distribution since these studies cause shear deformation of the whole skin using a pad.

This study develops the finger-mounted pin-array display which presents force distribution in various directions.

